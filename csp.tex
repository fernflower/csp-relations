\documentclass[12pt]{article}

\usepackage{graphicx}
\usepackage[utf8]{inputenc}
\usepackage[english,russian]{babel}
\usepackage[pdftex,unicode]{hyperref}
\usepackage[margin=15mm,left=2cm]{geometry}
\usepackage{indentfirst}
\usepackage{amsbib}
\usepackage{amsmath}

\sloppy
\clubpenalty=999
\widowpenalty=9999

\newtheorem{theorem}{Теорема}[section]
\newtheorem{lemma}[theorem]{Лемма}
\newtheorem{proposition}[theorem]{Утверждение}
\newtheorem{corollary}[theorem]{Следствие}

\newenvironment{proof}[1][Доказательство]{\begin{trivlist}
\item[\hskip \labelsep {\bfseries #1}]}{\end{trivlist}}
\newenvironment{definition}[1][Определение]{\begin{trivlist}
\item[\hskip \labelsep {\bfseries #1}]}{\end{trivlist}}
\newenvironment{example}[1][Example]{\begin{trivlist}
\item[\hskip \labelsep {\bfseries #1}]}{\end{trivlist}}
\newenvironment{remark}[1][Замечание]{\begin{trivlist}
\item[\hskip \labelsep {\bfseries #1}]}{\end{trivlist}}

\newcommand{\qed}{\nobreak \ifvmode \relax \else
      \ifdim\lastskip<1.5em \hskip-\lastskip
            \hskip1.5em plus0em minus0.5em \fi \nobreak
                  \vrule height0.75em width0.5em depth0.25em\fi}

% This is all for formatting and making the Table of Contents according to 
% spec. Don't play with it.
\makeatletter
\renewcommand\l@section[2]{%
    \ifnum \c@tocdepth >\z@
        \addpenalty\@secpenalty
        \addvspace{1.0em \@plus\p@}%
        \setlength\@tempdima{1.5em}%
        \begingroup
        \parindent \z@ \rightskip \@pnumwidth
        \parfillskip -\@pnumwidth
        \leavevmode \bfseries
        \advance\leftskip\@tempdima
        \hskip -\leftskip
#1\nobreak\ 
        \leaders\hbox{$\m@th\mkern \@dotsep mu\hbox{.}\mkern \@dotsep mu$}
    \hfil \nobreak\hb@xt@\@pnumwidth{\hss #2}\par
        \endgroup
    \fi}
\makeatother

\title{Критерий для класса предикатных схем}
\author{Василевская И.\,Ю.}
\date{\today}

\begin{document}
    \begin{titlepage}
        \begin{center}
            Московский Государственный Университет им. Ломоносова\\
            Факультет Вычислительной Математики и Кибернетики\\
            Кафедра Математической Кибернетики\\
            магистратура, отделение <<МММ СБИС>>\\[6cm]

            \large {Василевская Инесса Юрьевна}\\
            \LARGE \textbf {Верхние оценки одного параметра для класса предикатных схем}\\[0.8cm]
            \large \emph {Курсовая работа}\\[5.0cm]

            \begin{flushright}
                \large
                \begin{minipage}{0.40\textwidth}
                    \begin{flushleft}
                        \emph{Научный руководитель:}\\к.ф.м.н М.\,С.~Шуплецов
                    \end{flushleft}
                \end{minipage}
            \end{flushright}

            \vfill
            Москва\\
			2012
        \end{center}
    \end{titlepage}

\setcounter{page}{2}

\section{Введение}
\label{beginning}
В ряде работ (\cite{Shu09}, \cite{Shu11}) рассматривается задача синтеза для специального класса дискретных управляющих систем ~--
класса предикатных схем, который обобщает некоторые традиционные классы схем. Данные схемы строятся из предикатных элементов
и обладают рядом отличий от схем из других классов (например, направление протекания сигналов не является фиксированным).
Указанные выше работы в основном посвящены исследованию асимптотики функции Шеннона в классе предикатных схем,
 получению асимптотических оценок высокой степени точности и вопросам моделирования предикатными схемами схем из традиционных классов.

В силу особенностей класса предикатных схем, при его рассмотрении могут возникать вопросы,
нетипичные для традиционных классов. Так, к примеру, довольно любопытен вопрос функционирования предикатных схем с точки зрения оценки времени, которое нужно затратить для
вычисления конкретной схемы. 

\clearpage
\section{Основные определения}
Ниже приведены формальные определения используемых в работе понятий.

\subsection{Предикатные схемы}

В настоящей работе определение предикатных схем будет дано по аналогии с \cite{Shu11}.

\begin{definition}
\textit{Схемой из предикатных элементов} или \textit{предикатной схемой в базисе $\Pi$} назовем помеченный
неориентированный двудольный граф следующей структуры:

\begin{itemize}
\item каждая вершина из первой доли помечена некоторым множеством символов из алфавита X и/или 
множеством символов из алфавита Y. 
Алфавит $X$ соответствует \textit{входным} переменным предиката, а $Y$ ~-- его внутренним переменным, 
т.е. переменным, возникающим непосредственно в процессе вычисления; 

\item каждая вершина второй доли помечена некоторым символом $\pi_i$ из множества $\Pi$ и 
соединена $k$ ребрами, пронумерованными числами $1, ..., k$, с вершинами из первой доли.
\end{itemize}
\end{definition}

Вершины из первой доли будем называть \textit{узлами} схемы, а вершины из второй доли ~-- её \textit{предикатными элементами}. 
Узлы схемы, соединенные ребрами с предикатным элементом, будем называть полюсами этого элемента, 
а узлы, соответствующие входным переменным, ~-- полюсами схемы.
При этом считается, что узел является $j$-м полюсом предикатного элемента и соответствует 
его $j$-ой переменной, если соединяющее их ребро имеет номер $j$. Полюс схемы, которому приписано 
более одной входной переменной, называется кратным полюсом этой схемы. 

Будем считать элементарной такую предикатную схему, которая состоит либо из изолированной полюсной вершины, 
либо только из одного предикатного элемента $\pi_i$, $1 < i < k$, где $k$ ~-- число полюсов указанного элемента.

В тех случаях, когда это не вызывает разночтений, 
не будем различать узел схемы и переменную, 
символ которой приписан данной вершине, а также предикатный элемент и сам предикат, отвечающий этому элементу. 
Также, для удобства, в некоторых случаях будем использовать упрощенное описание предикатной схемы, 
опуская пометки дуг и некоторых внутренних вершин. 

\subsection{Функционирование предикатных схем}
Функционирование предикатного элемента с $k$ полюсами задается его характеристической функцией от $k$ переменных, 
связанных с указанными полюсами, и определяется тем, что предикатный элемент находится в допустимом состоянии на тех и 
только тех наборах значений этих переменных, на которых данная функция равна 1. 

Предикатная схема $\Sigma$ находится в допустимом состоянии на заданном наборе значений её полюсных переменных тогда и только тогда, 
когда существует такой набор значений внутренних переменных схемы, на котором все предикатные элементы схемы находятся в допустимых состояниях. 
Если же указанного набора значений внутренних переменных не существует, то считается, что схема находится в недопустимом состоянии на 
заданном наборе значений её полюсных переменных.

Предполагается, что предикатная схема $\Sigma$ реализует предикат $\pi$ от её полюсных переменных, если множество допустимых наборов $\pi$ 
совпадает с множеством тех наборов, на которых $\Sigma$ находится в допустимом состоянии. 
При этом схемы будем называть эквивалентными, если они реализуют равные предикаты. 
Отметим, что элементарная предикатная схема, состоящая из изолированной полюсной вершины, реализует тождественно истинный предикат.
 Будем считать также, что тождественно истинный (соответственно тождественно ложный)
 предикат реализуется любой предикатной схемой без входных полюсов, 
которая имеет непустое (соответственно пустое) множество допустимых состояний.

В общем случае граф предикатной схемы может содержать несколько компонент связности. 
В дальнейшем, будем считать, что схема не содержит компонент связности, 
которые не имеют полюсных узлов и для которых существует хотя бы один допустимый набор, так как такие компоненты не влияют на функционирование схемы.

\subsection{Операции над предикатными элементами}
В дальнейшем, предикат $\pi$, зависящий от n переменных, будем обозначать $\pi(x_1, \dots, x_n)$, а 
$\Pi(x_1, \dots, x_n)$ будет соответствовать множеству кортежей (наборов значений, множеству истинности) 
предиката $\pi(x_1, \dots, x_n)$.  

\begin{definition}
Суперпозицией двух предикатных схем, не имеющих общих вершин и пометок, 
будем называть их объединение с возможным отождествлением группы полюсов этих схем, 
которое сопровождается приписыванием новой (``объединенной'') вершине либо 
некоторого подмножества переменных данной группы, либо ``новой'' внутренней переменной.

При этом частными случаями суперпозиции являются следующие операции:

\begin{itemize}
    \item Суперпозиция без отождествления полюсов, или декартово произведение

    В статье \cite{Marchenkov} эта операция соответсвует операции конъюнкции.
    Конъюнкцией $\pi_1(x_1, \dots, x_m)$ и $\pi_2(x_1, \dots, x_n)$ называется такой $(m+n)$-местный предикат $\pi$, что
    $\pi(x_1, \dots, x_m, x_{m+1}, \dots, x_{m+n}) = \pi_1(x_1, \dots, x_m) \& \pi_2(x_{m+1}, \dots, x_{m+n})$.
    Причем множество кортежей $\Pi$ результирующего предиката $\pi$ равно $\Pi_1 \otimes \Pi_2$.

    \item Проекция или снятие полюсной переменной схемы

    По аналогии с \cite{Marchenkov}, проекцией предиката $\pi_1(x_1, \dots, x_n)$ по переменной $x_i$ называется такой
    $(n-1)$-местный предикат $\pi$, что:
    $\pi(x_1, \dots, x_{i-1}, x_{i+1}, \dots, x_n) = (\exists x_i) \pi_1(x_1, \dots, x_{i-1}, x_i, x_{i+1}, \dots, x_n)$.

    Сокращенно операцию проекции предиката $\pi$ по переменным $x_1, \dots, x_k$ будем обозначать $(\pi)\perp_{x_1, \dots, x_k}$

    \item Суперпозиция с отождествлением полюсов

    Суперпозицией $\pi_1(x_1, \dots, x_m)$ и $\pi_2(y_1, \dots, y_n)$ с отождествлением по первым k переменным 
    называется $(m+n)$-местный предикат
    $\pi(x_1, \dots, x_k, x_{k+1}, \dots, x_m, y_{m+1}, \dots, y_n) = \pi_1(x_1, \dots, x_k, x_{k+1} \dots, x_m) \& \pi_2(x_1, \dots, x_k, y_{k+1}, \dots, y_n)$.

    Сокращенно операцию суперпозиции предикатов $\pi_1$ и $\pi_2$ с отождествлением полюсов 
    $x_1, \dots, x_k$ и $y_1, \dots, y_k$ соответственно, будем обозначать $(\pi_1|\pi_2)_{x_1=y_1, \dots, x_k=y_k}$.

    \item Отождествление двух полюсных переменных схемы

    $\pi(x_1, \dots, x_n) = \pi_1(x_1, x_1, \dots, x_n)$

    Сокращенно операцию отождествления двух переменных $x_i, x_j$ предиката $\pi$ будем обозначать 
    $(\pi)|_{x_i=x_j}$

    \item Подстановка констант $(\sigma_1, \dots, \sigma_k)$ вместо первых k переменных

    $\pi(x_1, \dots, x_n) = \pi_1(\sigma_1, \dots, \sigma_k, x_{k+1}, \dots, x_n)$. 
    Сокращенно операцию подстановки констант вместо переменных $x_1, \dots, x_k$ предиката $\pi$
    будем обозначать $(\pi)|_{x_1=\sigma_1, \dots, x_k=\sigma_k}$.

\end{itemize}
\end{definition}
\subsection{Сведение модели к обобщенной задаче выполнимости}

\begin{definition}
Обобщенная задача выполнимости (CSP) \footnote{Constraint Satisfaction Problem} 
~--- это тройка $<V,D,C>$, где $V$ ~-- набор переменных, $D$ ~-- область определения (домен допустимых значений), 
а $C$ ~-- набор ограничений. Ограничения имеют вид $<t, R_n>$, где $t$ ~--- кортеж из $n$ переменных множества $V$, 
$R_n$ ~-- заданное на $D$ отношение арности $n$. 
Решением задачи CSP является набор значений переменных $V$, который удовлетворяет всем ограничениям из $C$. 
\end{definition}

Так, например, задачу 3-ВЫП можно рассматривать как обобщенную задачу выполнимости над булевым доменом, с 
состоящим лишь из отношений-конъюнкций множеством ограничений.

В ряде работ рассматривалась вычислительная сложность задачи CSP. Несмотря на то, что в общем случае задача CSP
принадлежит классу NP, при введении определенных ограничений на множество $C$ или структуру так называемого
графа ограничений\footnote{constraints graph}, существуют быстрые алгоритмы нахождения решения. 
Так, например, в \cite{Shaeffer78} показано, 
что, в тех случаях, когда множество $C$ представляет собой дерево, существуют полиномиальные алгоритмы нахождения решения.

\subsection{Сведение задачи вычисления предикатной схемы к задаче нахождения решения задачи CSP}

Пусть предикатная схема $\Sigma$ имеет $m$ входных переменных $x_1, \ldots , x_m$, 
$k$ внутренних переменных $y_1, \ldots , y_k$ и $n$ предикатных элементов $B = \pi_1, \dots , \pi_n$. 
Тогда, для определения функционирования этой схемы нужно найти допустимые значения внутренних переменных на всех $2^{m}$
наборах входных переменных. 

Нетрудно видеть, что эта задача может быть сведена к $2^m$ задачам CSP $<V, D, C>$ следующим образом:
$V$ = $Y$, $D = [0, 1]$, $C = B$. Перед началом вычисления каждой из задач, переменные $x_1, \ldots , x_m$ 
во всех ограничениях, их содержащих, отождествляются с соответствующими константами $\alpha_1, \ldots , \alpha_m$.

В обеих моделях, сложность вычислений в худшем случае, при применении алгоритма поиска 
с откатом без памяти, требует $2^m * (2^k)^N$ операций, где $N$ ~-- количество предикатов (ограничений) 
в схеме (графе ограничений).

\subsection{Шефферов базис}

\begin{definition} ФАЛ
$f (x_1, \ldots, x_n)$ от $n$ переменных сохраняет предикат $ (x_1, \ldots, x_m)$ от $m$
переменных тогда и только тогда, когда для любых $n$ допустимых наборов $\alpha^i = (\alpha_1^i, \ldots, \alpha_n^i), 
i = \overline{1, n}$ значений переменных предиката $\pi$ набор
значений $( f(\alpha_1^1, \ldots, \alpha_1^n), \ldots, f(\alpha_m^1, \ldots, \alpha_m^n) )$
тоже является допустимым для предиката $\pi$.
\end{definition}

Определим следующие классы:
\begin{itemize}
    \item{$T_0$}~-- множество всех предикатов, сохраняющих константную ФАЛ 0.
    \item{$T_1$}~-- множество всех предикатов, сохраняющих константную ФАЛ 1.
    \item{S}~-- множество всех предикатов, сохраняющих ФАЛ $\bar{x}$.
    \item{K}~-- множество всех предикатов, сохраняющих все конъюнкции от двух и более переменных.
    \item{D}~-- множество всех предикатов, сохраняющих все дизъюнкции от двух и более переменных.
    \item{SM}~-- множество всех предикатов, сохраняющих все монотонные самодвойственные ФАЛ.
    \item{SL}~-- множество всех предикатов, сохраняющих все монотонные линейные ФАЛ.
\end{itemize}

\begin{theorem}
\textbf{(Критерий полноты для предикатных схем)}. Система из $B$ предикатов является полной $\iff$
она не лежит целиком ни в одном из 7 предполных классов: $T_0, T_1, SM, SL, S, K, D$. \cite{Shu11} 
\end{theorem}

\begin{remark}
Если P~-- предполный класс, то P замкнут относительно операций подстановки констант,
декартового произведения, проекции.
\end{remark}


\begin{definition}
Шефферовым базисом будем называть полный базис, состоящий из одного предиката. Сам этот предикат
будем называть шефферовым предикатом.
\end{definition}

\begin{definition}
Минимальным предикатом $\pi_{min}(x_1, \dots, x_n) \notin P$, где $P$~-- один из предполных классов, будем называть такой 
предикат $\pi$, что $\forall \pi'(x_1, \dots, x_{n-1}) = \pi_{min} \perp_{x_i} \implies \pi' \in P$. Т.е. любой предикат,
полученный из $\pi_{min}$ проекцией по одной из переменных предиката, принадлежит $P$.
\end{definition}

\begin{definition}
Будем говорить, что предикат $\pi_1$ входит в предикат $\pi_2$, или что предикат $\pi_2$ является
расширением предиката $\pi_1$, если множество кортежей $\Pi_1$ предиката $\pi_1$ 
содержится во множестве кортежей $\Pi_2$ предиката $\pi_2$.
\end{definition}

\begin{definition}
Набор $(\alpha_1, \dots, \alpha_n)$ называется существенным для предиката $\pi$ арности $n$, если
$\pi(\alpha_1, \dots, \alpha_n) = 0$ и существует $b_1, \dots, b_n$ такие, что для любого $i \in \{ 1, 2, \dots, n\}$
выполняется 
$\pi(\alpha_1, \dots, \alpha_{i-1}, b_i, a_{i+1}, \dots, a_n)$ = 1.
\end{definition}

\begin{definition}
Предикат арности $n$ называется существенным, если он не может быть представлен в виде конъюнкции
предикатов меньшей арности.
\end{definition}

\begin{lemma}
Следующие три условия эквиваленты: 
\begin{enumerate}
\item предикат $\pi$ является существенным;
\item $\pi \neq \pi_{negate}(x_1, x_2)$, где $\Pi_{negate} = \{ (01), (10) \}$;
\item для предиката $\pi$ существует существенный набор. 
\end{enumerate}\cite{Zhuk}
\end{lemma}

\begin{definition}
Рассмотрим предикат 
$\pi$, множество кортежей которого имеет вид $\Pi = \{ (\alpha, \bar{\alpha}, \dots, \bar{\alpha}), (\bar{\alpha}, \alpha, \dots, \bar{\alpha}), \dots, $
$(\bar{\alpha}, \bar{\alpha}, \dots, \alpha) \}$. Нетрудно видеть, что $\pi$~-- существенный предикат, 
$(\bar{\alpha}, \bar{\alpha}, \dots, \bar{\alpha})$~-- существенный набор. 
В дальнейшем, предикаты, множество кортежей которых обладает такой структурой, будем называть 
симметричными предикатами порядка n и обозначать $\pi_{symm}^n$. Нетрудно видеть, что существенный предикат является 
расширением симметричного предиката.
\end{definition}

\clearpage
\section{Основная часть}


\begin{definition}
Предикатная схема, которую можно уложить на плоскости без реберных пересечений таким образом, что полюса предикатов, 
соответствующие внешним переменным, лежат на внешней границе, называется планарной. 
\end{definition}

\begin{definition}
Базис называется планарным, если в нем можно реализовать любой предикат планарной схемой.
\end{definition}

\begin{theorem}
\label{ZarankTheorem}
(Заранкевича) Число скрещиваний полного двудольного $K_{n,m}$ графа $\leq$
$\lfloor \frac{n}{2} \rfloor \lfloor \frac{n-1}{2} \rfloor \lfloor \frac{m}{2} \rfloor \lfloor \frac{m-1}{2} \rfloor$.
\cite{Zarank54}
\end{theorem}

Заметим, что операции отождествления не более чем по двум переменным ($|_{x_1, x_2}$), проекции ($\perp$),
подстановки констант ($|_{\sigma_1, \dots, \sigma_k}$) и отождествления входов сохраняют планарность.
В дальнейшем операции над предикатными элементами, сохраняющие планарность схемы,
будем называть операциями планарной суперпозиции.

Расширим набор доступных операций планарной суперпозиции операцией отрицания $i$-той переменной ($\neg_i$ или просто $\neg$). 
Правомерность такого расширения будет показана ниже.
Эту операцию можно определить как конъюнкцию исходного предиката $\pi_1(x_1, \dots, x_n)$ и предиката
$\pi_{negate}(y_1, y_2)$, где $\Pi_{negate} = \{ (01), (10) \}$:

$\pi(x_1, \dots, y, \dots, x_k) = \pi_1(x_1, \dots, x_i, \dots, x_k) \& \pi_{negate}(x_i, y)$

Операция взятия отрицания нескольких переменных $x_i, x_j, x_t$ предиката $\pi_1$
определяется как последовательное применение операций отрицания одной переменной:

$\pi(x_1, \dots, x_k) = (\neg_t (\neg_j (\neg_i \pi_1)))$

Покажем, что, применяя над полным базисом $B$ только операции планарной суперпозиции ~-- $\perp$, $|_{x_1, x_2}$,
и отождествления входов, можно получить предикаты-константы
$\pi_0(x_1), \Pi_0=(0)$ и $\pi_1(x_1), \Pi_1=(1)$ и предикат-отрицание $\pi_{negate}(x_1, x_2), \Pi_{neg}=( (01), (10) )$.

\subsection{Получение констант}
\begin{lemma}
\label{eq:const}
Пусть дан полный базис B. Тогда, применением операции отождествления входов к предикатам из B,
можно получить константу.
\end{lemma}

\begin{proof}
$\exists \pi \in B, \pi \notin S \Rightarrow \exists \alpha_1 \in \Pi, \bar{\alpha_1} \notin \Pi$
Отождествляя все переменные, равные 0, и все переменные, равные 1, получаем одну из констант.
Если получили константу 0, то берем $\pi_1 \notin T_0 \Rightarrow \forall \beta_1 \in \Pi_1, \beta_1=(0 \ldots 0)$. 
Подставляем вместо определенных переменных 0, получаем константу 1. 
В случае константы 1, поступаем аналогично, $\bigtriangleup$.
\end{proof}
\begin{figure}[htb]
\centering
\includegraphics[width=0.01\textwidth]{3_2to3.png}
\caption{Получение константы}
\label{fig:constant}
\end{figure}

\subsection{Получение отрицания}
\begin{lemma}
\label{eq:negate}
Пусть дан полный базис B. Тогда, применением операций подстановки констант
и $\perp$ к предикатам из B, можно получить предикат $\pi_{negate}$.
\end{lemma}

\begin{proof}
$\exists \pi \in B, \pi \notin K \Rightarrow \exists \alpha_1, \alpha_2 \in \Pi$,
$\alpha_1\&\alpha_2=\beta, \beta \notin \Pi$, причем эти наборы должны отличаться как минимум по двум переменным $x_i, x_{j}$.
Таким образом, подставив вместо всех общих переменных этих наборов соответствующие константы 0 и 1, применив операцию 
$\perp_{\forall x_t, t \neq i,j}$
(чтобы оставить в точности две необходимых для сохранения свойства $\pi \notin K$ переменных $x_i, x_j$), 
получим предикат-отрицание $\pi_{negate}, \bigtriangleup$.
\begin{figure}[htb]
\centering
\includegraphics[width=0.01\textwidth]{3_2to3.png}
\caption{Получение $\pi_{neg}$}
\label{fig:negation}
\end{figure}

\end{proof}

\begin{corollary}
Обосновали добавление операции отрицания i-той переменной $\neg_i$ к множеству стандартных операций планарной суперпозиции.
\end{corollary}

\subsection{Обоснование планарности}
\subsubsection{Планарные базисы}

\begin{figure}[htb]
\centering
\includegraphics[width=0.5\textwidth]{scheff3.png}
\caption{Шефферовы предикаты от 3 переменных}
\label{fig:sheff}
\end{figure}

\label{planar_basis}
\begin{lemma}
\label{eq:planar_algo}
Существует алгоритм преобразования непланарной предикатной схемы $\Sigma$ в базисе $B=\{\pi_{symm}^3\}$
в планарную предикатную схему $\Sigma'$ в базисе B, добавляющий не более, чем 
$5 * n^2*(n-1)^2 $ предикатных элементов, где n ~-- количество предикатов в схеме $\Sigma$.
\end{lemma}
\begin{proof}
Так как в базисе $B=\{\pi_{symm}^3\}$ можно планарной схемой реализовать предикат 
$\pi_{linear}, \Pi_{linear} = \{ (001), (010), (100), (111) \}$, то замещая каждое реберное пересечение в схеме на 
3 предиката ($\pi_{linear}$) по схеме на рис. \ref{fig:xor}, получаем планарную реализацию требуемого предиката
в базисе B.

Если $k$~-- количество реберных пересечений то, по теореме \ref{ZarankTheorem}, 
$k \leq \lfloor \frac{n}{2} \rfloor \lfloor \frac{n-1}{2} \rfloor \lfloor \frac{3n}{2} \rfloor \lfloor \frac{3n-1}{2} \rfloor \le \lfloor \frac{9}{16} * n^2*(n-1)^2 \rfloor$.
Так как для существует реализации $\pi_{linear}$ в B, требующая 3 симметричных предиката, а для избавления
от реберного перечения понадобится 3 предиката $\pi_{linear}$, получаем, 
что суммарное число предикатов увеличивается на
$3*3*k \leq \lfloor \frac{81}{16} * n^2 * (n-1)^2 \rfloor = 5 * n^2 * (n-1)^2$, $\bigtriangleup$.
\end{proof}

Лемму \ref{eq:planar_algo} можно естественным образом обобщить:
\begin{lemma}
\label{general_planar_algo_complexity}
Существует алгоритм преобразования непланарной предикатной схемы $\Sigma$ в базисе $B = \{\pi_1, \dots, \pi_m \}$
в планарную предикатную схему $\Sigma'$ в базисе B, добавляющий не более, чем $\lfloor \frac{k^2}{16} * 3 * L_{\pi_{linear}} * n^2*(n-1)^2 \rfloor$ 
предикатных элементов, где $k$~- максимальная арность предикатов из B, n~-- число предикатов в схеме $\Sigma$, 
$L_{\pi_{linear}}$~-- сложность планарной реализации $\pi_{linear}$.
\end{lemma}

\begin{figure}[htb]
\centering
\includegraphics[width=0.5\textwidth]{intersection.png}
\caption{Получение планарной схемы}
\label{fig:xor}
\end{figure}


\begin{lemma}
\label{eq:lemma_sm}
Если $\pi \notin SM$, то, применяя операции $\perp$ и $\neg$,
можно получить $\pi_{min} \notin SM$, причем $\pi_{min}$ будет являться расширением симметричного предиката.
\end{lemma}

\begin{proof}
Рассмотрим $\pi_{min} \notin SM$ такой, что $\forall i, \pi_1 = \pi_{min} \perp_{x_i}$,
$\pi_1 \in SM$.
Пусть $\beta = (\beta_1, \dots, \beta_k)$~-- набор, $\notin \pi_{min}$, полученный в результате применения какой-то несамодвойственной функции.
Заметим, что $\beta \neq \forall \alpha \in \pi$. Однако, для соблюдения верхнего условия,
$\Pi_{min}$ должно содержать наборы 
$(\bar{\beta_1}, \beta_2, \dots, \beta_k), (\beta_1, \bar{\beta_2}, \dots, \beta_k), \dots, (\beta_1, \beta_2, \dots, \bar{\beta_k})$.

Нетрудно видеть, что набор $(\bar{\beta_1}, \bar{\beta_2}, \dots, \bar{\beta_k})$~-- существенный для $\pi_{min}$.

Так как класс $SM$ замкнут относительно $\neg$, применяя эту операцию к $\pi_{min}$ можно получить $\pi'_{min}$,
такой что набор $(\sigma, \sigma, \dots, \sigma)$ является для него существенным, а сам $\pi'_{min}$ является расширением 
симметричного предиката, $\bigtriangleup$.
\end{proof}

\begin{corollary}
\label{lemma_sm_corollary}
$\pi \notin SM \Longrightarrow \exists \pi_{min} \notin SM, \exists 
\alpha = (\sigma, \sigma, \dots, \sigma) \notin \Pi_{min}, \sigma \in \{0, 1\}$.
\end{corollary}

\begin{lemma}
\label{eq:super_new}
Пуcть $\pi \notin SM$, $\pi$~-- расширение симметричного предиката $\pi_{symm}^3$. 
Тогда, применяя операции планарной суперпозиции, можно получить планарную реализацию $\pi_{linear}$.
\end{lemma}

\begin{proof}
По условию, $\Pi = \Pi_{symm}^3 \bigcup A$, где
$ A \subseteq(\alpha \bar{\alpha} \alpha), (\alpha \alpha \bar{\alpha}), (\alpha \bar{\alpha} \alpha), (\alpha \alpha \alpha) \} $.

Заметим, что во всех наборах из A от 2 до 3 координат равны $\alpha$, и $|A| = k, 1 \leq k \leq 4$.
Таким образом, примененяя операцию суперпозиции с отождествлением по 2 переменным
($(\pi|\pi_1)_{x_{i_1}=y_1, x_{i_2} =y_2}$) предиката $\pi(x_1, x_2, x_3)$ 
и предиката $\pi_1(y_1, y_2)$ с множеством наборов $\Pi_1 = \{ (\bar{\alpha}, \bar{\alpha}), (\bar{\alpha}, \alpha), (\alpha, \bar{\alpha})\}$,
где $i_1 \neq i_2 \neq j$,
\[ j = \left \{
  \begin{array}{l l}
     t & \quad \text{если $\exists \widetilde{\beta}=(\beta_1, \beta_2, \beta_3) \in A, \beta_t=\bar{\alpha}$ }\\
     \forall t \in [1, 3] & \quad \text{если $A = \{ (\alpha, \alpha, \alpha)\}$}
            \end{array} \right. \]
не более чем k раз, можно убрать ``лишние'' наборы и получить симметричный предикат.

Остается получить предикат $\pi_1(x_1, x_2)$.
Если $\pi_1(x_1, x_2)$ не получается из $\pi(x_1, x_2, x_3)$ проекцией по какой-либо переменной, то рассмотрим 2 случая. 

1. Пусть $A \neq \{ (\alpha, \alpha, \alpha) \}$. Тогда
$\exists! \widetilde{\beta}=(\beta_1, \beta_2, \beta_3) \in A, \beta_t=\bar{\alpha}$. 
Подставив на место $t$-ой переменной
предиката $\pi$ константу $\bar{\alpha}$, и спроецировав по переменной t получим: 
$(\pi(x_1, x_2, x_3)|_{x_t=\bar{\alpha}} \perp_{x_t} )$ = 
$\pi_1'(x_1, x_2), \Pi_1'=\{(\bar{\alpha}, \alpha), (\alpha, \bar{\alpha}), (\alpha, \alpha)\}$, из которого искомый предикат
$\pi_1$ получается применением операции $\neg$.

В базисе из симметричного предиката $\pi_{symm}^3$ предикат $\pi_{linear}$ получается известным образом, с использованием 3
предикатов $\pi_{symm}^3$.

2. Пусть $A = \{ (\alpha, \alpha, \alpha) \}$. Тогда $\pi=\pi_{linear}, \bigtriangleup$
\end{proof}

\begin{corollary}
\label{complexity}
Сложность получения предиката $\pi_{linear}$ из $\pi(x_1, x_2, x_3) \notin SM \leq 3$.
\end{corollary}
\begin{proof}
$\pi_{\bar{\alpha}}$, в худшем случае, реализуется с использованием 2 предикатов; $\pi_{negate}$~-- с использованием 1 
$\pi_{\alpha}$. Для получения предиката  $\pi_1(x_1, x_2)$ из $\pi'_1(x_1, x_2)$, в худшем случае может понадобится 2 отрицания.
Итого $1 + 2 * 1 = 3$ предиката, $\bigtriangleup$.

\end{proof}

Следующие две леммы показывают как, применяя операции планарной суперпозиции, от симметричного предиката
размерности n (его расширения) перейти к симметричному предикаты размерности $n-1$ (его расширению).

\begin{lemma}
\label{eq:svedenie1}
Если $\pi(x_1, \dots, x_n) = \pi_{symm}^n, n > 3$, то, 
применяя операции подстановки констант и $\perp$, можно получить $\pi_{symm}^{n-1}$.
\end{lemma}

\begin{proof}
Так как 
$\pi \notin SM, \exists \widetilde{\alpha} = (\alpha, \dots, \alpha)$, $\pi$ не сохраняет $\widetilde{\alpha}$.
Тогда 
$\pi_1 = ( (\pi|_{x_1=\alpha}) \perp_{x_1} )$, где $\pi_1(x_1, \dots, x_{n-1}) = \pi_{symm}^{n-1}, \bigtriangleup$
\end{proof}

\begin{lemma}
\label{eq:svedenie2}
Если $\pi(x_1, \dots, x_n) \notin SM, n > 3$, то, применяя операции подстановки констант, $\neg$ и $\perp$,
можно получить $\pi_1(x_1, \dots, x_{n-1}) \notin SM, \bigtriangleup$.
\end{lemma}

\begin{proof}
По следствию \ref{lemma_sm_corollary}, 
$\exists \pi_{min} \notin SM, \alpha=(\sigma, \dots, \sigma) \notin \Pi_{min}$. Тогда предикат $\pi_1$, такой что
$\pi_1 = ( (\pi_{min}|_{x_1=\sigma}) \perp_{x_1} )$, не принадлежит $SM$, и зависит от $n-1$ переменной, $\bigtriangleup$.
\end{proof}
%\label{eq:3tuples}
%\textbf{Следствие.} Пусть $\pi(x_1, x_2, x_3) \notin SM$. 
%Тогда : $\pi_1 = (((\pi SELECT x_1=\bar{\alpha}) PROJECT x_1)$ и 
%$\Pi_1 = \{ (\bar{\alpha}\alpha), (\alpha\bar{\alpha}), (\alpha\alpha) \}$

\subsubsection{Алгоритм планарного сведения}

\begin{theorem}
\label{Theo1}
Пусть дана непланарная предикатная схема $\Sigma$ в полном предикатном базисе $B$. 
Тогда из $\Sigma$, применением операций планарной суперпозиции, можно получить схему $\Sigma'$,
реализующую тот же предикат, что и $\Sigma$, но являющуюсь планарной.
\end{theorem}
\begin{proof}
Приведем конструктивный алгоритм преобразования исходной схемы.

Так как $B$~-- полный базис, то по леммам \ref{eq:const} и \ref{eq:negate}, применением ограниченного набора операций планарной
суперпозиции, можно получить планарную реализацию предикатов $\pi_0, \pi_1$ и $\pi_{negate}$. 
Таким образом, становится доступным весь набор операций планарной суперпозиции. 

Далее заменяем каждое реберное пересечение по лемме \ref{eq:planar_algo}. 

Полученная в результате вышеописанных преобразований схема является планарной, $\bigtriangleup$.
\end{proof}
\begin{corollary}
Сложность преобразования непланарной предикатной схемы $\Sigma$ в базисе B, 
все элементы которого зависят не более, чем от 3 переменных, 
в планарную схему $\Sigma'$ в B не превышает $5 * n^2 * (n-1)^2$, где $n$~--число предикатов в $\Sigma$.
\end{corollary}
\begin{proof}
Оценим сложность каждого шага. Сложность получения планарной реализации предиката $\pi_{linear}$, по лемме \ref{complexity},
не превышает 3. Сложность избавления от реберных пересечений, по лемме \ref{general_planar_algo_complexity}, не 
превысит в таком случае $\lfloor \frac{3^2}{16} * 3 * 3 * n^2 * (n-1)^2 \rfloor \le 5 * n^2 * (n-1)^2, \bigtriangleup$.
\end{proof}

\begin{theorem}
Пусть дана непланарная предикатная схема $\Sigma$ в полном предикатном базисе $B$. 
Тогда из $\Sigma$, применением операций планарной суперпозиции, можно получить схему $\Sigma'$,
реализующую тот же предикат, что и $\Sigma$, но являющуюсь планарной.
\end{theorem}

\begin{proof}
Предикаты $\pi_0, \pi_1, \pi_{negate}$ получаем аналогично доказательству теоремы \ref{Theo1}.

Выделяем из базиса B $\pi(x_1, \dots, x_n) \notin SM$. По леммам \ref{eq:svedenie1} и \ref{eq:svedenie2}, 
применяя операции планарной суперпозиции к предикату $\pi(x_1, \dots, x_n)$, можно получить 
предикат $\pi'(x_1, x_2, x_3)$, являющийся либо симметричным предикатом от 3 переменных, либо его расширением. 

Далее, по лемме $\ref{eq:super_new}$ из предиката $\pi'$ получаем $\pi_{linear}$ и преобразовываем схему в планарную
по известному алгоритму, $\bigtriangleup$.
\end{proof}
\begin{corollary}
Сложность преобразования непланарной предикатной схемы $\Sigma$ в базисе B, 
все элементы которого зависят не более, чем от k переменных, 
в планарную схему $\Sigma'$ в B не превышает $\lfloor \frac{18 * k^2 * (k-2)}{16} * n^2 * (n-1)^2 \rfloor$, где n~-- число предикатов в схеме $\Sigma$.
\end{corollary}
\begin{proof}
Оценим сложность получения несамодвойственного предиката $\pi_1(x_1, x_2, x_3)$ из несамодвойственного предиката
$\pi(x_1, \dots, x_t)$. По леммам \ref{eq:svedenie1} и \ref{eq:svedenie2}, в худшем случае придется применить $k-2$ 
операции подстановки констант, реализация каждой из которых требует не более чем 2 предиката.
Тогда сложность получения $\pi_{linear}$, по лемме \ref{eq:super_new}, составит $2 * (k-2) * 3$, а
результирующая сложность не превысит $\lfloor \frac{6 * (k-2)}{16} * k^2 * 3 * n^2 * (n-1)^2 \rfloor $ по лемме 
\ref{general_planar_algo_complexity}.
\end{proof}

\clearpage
\subsubsection{Примеры планарных сведений некоторых предикатов}

\begin{figure}[htb]
\centering
\includegraphics[width=1.0\textwidth]{3_2to3.png}
\caption{Сведение $\pi_5^3$ к $\pi_{symm}^3$ }
\label{fig:3_2to3}
\end{figure}

\begin{figure}[htb]
 \centering
\includegraphics[width=0.6\textwidth]{4to3.png}
\caption{Сведение $\pi_{symm}^4$ к $\pi_{symm}^3$ }
\label{fig:4to3}
\end{figure}

%\begin{figure}[htb]
%\centering
%\includegraphics[width=1.0\textwidth]{3_3to3_2.png}
%\caption{Сведение $\pi_{symm+3}^3$ к $\pi_{symm+2}^3$}
%\label{fig:3_3to3_2}
%\end{figure}

%\begin{figure}[htb]
%\centering
%\includegraphics[width=1.0\textwidth]{3_4to3_2.png}
%\caption{Сведение $\pi_{symm+4}^3$ к $\pi_{symm+2}$}
%\label{fig:3_4to3_2}
%\end{figure}

\clearpage
\subsection{Получение минимальных схем в базисе $\pi_{symm}^3$}
В Теореме 1 было показан алгоритм планарного сведения произвольного полного базиса к $\pi_{symm}^3$. 
В этой главе будут рассмотрены вопрос синтеза планарных предикатных схем по заданной функции и 
приведены примеры минимальных планарных схем для некоторых функций.


\begin{figure}[htb]
\centering
\includegraphics[width=0.7\textwidth]{min_and.png}
\caption{Схема для $\pi_{and}$}
\label{fig:and}
\end{figure}

% Список цитируемой литературы
\clearpage
\addcontentsline{toc}{section}{Список цитируемой литературы}
\thebibliography{99}
\RBibitem{Shu09}
    \by М.~С.~Шуплецов
    \paper Оценки высокой степени точности для сложности предикатных схем в~некоторых базисах
    \inbook Физико-математические науки
    \serial Уч\"eн. зап. Казан. гос. ун-та. Сер. Физ.-матем. науки
    \yr 2009
    \vol 151
    \issue 2
    \pages 173--184
    \publ Изд-во Казанского ун-та
    \publaddr Казань
    \mathnet{http://mi.mathnet.ru/uzku760}

\bibitem{Shu11}Методы синтеза и оценки сложности схем, построенных из элементов предикатного типа, диссертация

\bibitem{CSP10} Handbook of Constraint Programming, ISBN 9780444527264; 2010 г.

\bibitem{Shaeffer78} Schaefer, Thomas J. (1978). 
``The Complexity of Satisfiability Problems''. STOC 1978. pp. 216–226. doi:10.1145/800133.804350.

\bibitem{Zarank54} Zarankiewicz, K. "On a Problem of P. Turán Concerning Graphs." Fund. Math. 41, 137-145, 1954. 

\bibitem{Prosc89} Arnborg, S.; Proskurowski, A. (1989), 
``Linear time algorithms for NP-hard problems restricted to partial k-trees'',
Discrete Applied Mathematics 23 (1): 11–24, doi:10.1016/0166-218X(89)90031-0.

\bibitem{Gott10} 
Georg Gottlob, Reinhard Pichler, and Fang Wei. 2010. Bounded treewidth as a key to tractability of knowledge representation and reasoning. Artif. Intell. 174, 1 (January 2010), 105-132. DOI=10.1016/j.artint.2009.10.003 http://dx.doi.org/10.1016/j.artint.2009.10.003

\bibitem{Diestel00}
Diestel, Reinhard (2000), Graph Theory, Graduate Texts in Mathematics, 
173, Springer-Verlag, ISBN 0-387-98976-5.

\bibitem{Boedlander96}
H. L. Bodlaender, A linear-time algorithm for finding 
tree-decompositions of small
treewidth, SIAM J. Comput. 25 (1996), 1305–1317

\bibitem{Marchenkov}
C.C. Марченков, ``Предполнота замкнутых классов в $P_k$: предикатный подход'', Проблемы Кибернетики

\bibitem{Zhuk}
Жук, монография

\endthebibliography

\end{document}
