\documentclass[14pt,fleqn,russian]{extarticle}

\usepackage[utf8]{inputenc}
\usepackage[russian]{babel}
\usepackage[pdftex,unicode]{hyperref}
\usepackage[margin=15mm,left=2cm]{geometry}
\usepackage{hyperref}
\usepackage{indentfirst}

\sloppy
\clubpenalty=999
\widowpenalty=9999

% This is all for formatting and making the Table of Contents according to 
% spec. Don't play with it.
\makeatletter
\renewcommand\l@section[2]{%
    \ifnum \c@tocdepth >\z@
        \addpenalty\@secpenalty
        \addvspace{1.0em \@plus\p@}%
        \setlength\@tempdima{1.5em}%
        \begingroup
        \parindent \z@ \rightskip \@pnumwidth
        \parfillskip -\@pnumwidth
        \leavevmode \bfseries
        \advance\leftskip\@tempdima
        \hskip -\leftskip
#1\nobreak\ 
        \leaders\hbox{$\m@th\mkern \@dotsep mu\hbox{.}\mkern \@dotsep mu$}
    \hfil \nobreak\hb@xt@\@pnumwidth{\hss #2}\par
        \endgroup
    \fi}
\makeatother

\title{Верхние оценки параметра treewidth для класса предикатных схем}
\author{Василевская И.\,Ю.}
\date{\today}

\begin{document}
    \begin{titlepage}
        \begin{center}
            Московский Государственный Университет им. Ломоносова\\
            Факультет Вычислительной Математики и Кибернетики\\
            Кафедра Математической Кибернетики\\
            магистратура, отделение <<МММ СБИС>>\\[6cm]

            \large {Василевская Инесса Юрьевна}\\
            \LARGE \textbf {Верхние оценки параметра treewidth для класса предикатных схем}\\[0.8cm]
            \large \emph {Курсовая работа}\\[5.0cm]

            \begin{flushright}
                \large
                \begin{minipage}{0.40\textwidth}
                    \begin{flushleft}
                        \emph{Научный руководитель:}\\к.ф.м.н М.\,С.~Шуплецов
                    \end{flushleft}
                \end{minipage}
            \end{flushright}

            \vfill
            Москва\\
			2012
        \end{center}
    \end{titlepage}

\setcounter{page}{2}

\begin{thebibliography}{99}

\end{thebibliography}


\end{document}
