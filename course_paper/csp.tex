\documentclass[12pt]{article}

\usepackage[utf8]{inputenc}
\usepackage[english,russian]{babel}
\usepackage[pdftex,unicode]{hyperref}
\usepackage[margin=15mm,left=2cm]{geometry}
\usepackage{indentfirst}
\usepackage{amsbib}

\sloppy
\clubpenalty=999
\widowpenalty=9999

% This is all for formatting and making the Table of Contents according to 
% spec. Don't play with it.
\makeatletter
\renewcommand\l@section[2]{%
    \ifnum \c@tocdepth >\z@
        \addpenalty\@secpenalty
        \addvspace{1.0em \@plus\p@}%
        \setlength\@tempdima{1.5em}%
        \begingroup
        \parindent \z@ \rightskip \@pnumwidth
        \parfillskip -\@pnumwidth
        \leavevmode \bfseries
        \advance\leftskip\@tempdima
        \hskip -\leftskip
#1\nobreak\ 
        \leaders\hbox{$\m@th\mkern \@dotsep mu\hbox{.}\mkern \@dotsep mu$}
    \hfil \nobreak\hb@xt@\@pnumwidth{\hss #2}\par
        \endgroup
    \fi}
\makeatother

\title{Верхние оценки параметра treewidth для класса предикатных схем}
\author{Василевская И.\,Ю.}
\date{\today}

\begin{document}
    \begin{titlepage}
        \begin{center}
            Московский Государственный Университет им. Ломоносова\\
            Факультет Вычислительной Математики и Кибернетики\\
            Кафедра Математической Кибернетики\\
            магистратура, отделение <<МММ СБИС>>\\[6cm]

            \large {Василевская Инесса Юрьевна}\\
            \LARGE \textbf {Верхние оценки одного параметра для класса предикатных схем}\\[0.8cm]
            \large \emph {Курсовая работа}\\[5.0cm]

            \begin{flushright}
                \large
                \begin{minipage}{0.40\textwidth}
                    \begin{flushleft}
                        \emph{Научный руководитель:}\\к.ф.м.н М.\,С.~Шуплецов
                    \end{flushleft}
                \end{minipage}
            \end{flushright}

            \vfill
            Москва\\
			2012
        \end{center}
    \end{titlepage}

\setcounter{page}{2}

\section{Введение}
В ряде работ (\cite{Shu09}, что-то) рассматривается задача синтеза для специального класса дискретных управляющих систем ~--
класса предикатных схем, который обобщает некоторые традиционные классы схем. Данные схемы строятся из предикатных элементов
и обладают рядом отличий от схем из других классов (например, направление протекания сигналов не является фиксированным).

Настоящая работа посвящена изучению так называемого параметра \textit{ширины декомпозиции} 
\footnote
{В западной литературе известного как treewidth. В силу отсутствия общепринятого перевода, автор вводит 
понятие <<ширины декомпозиции>>.} предикатных схем в некотором базисе.

\subsection{Основные определения}
Ниже приведены формальные определения используемых в работе понятий.

\subsubsection*{Предикатные схемы}

В настоящей работе определение предикатных схем будет дано по аналогии с \cite{Shu11}.

\textit{Схемой из предикатных элементов} или \textit{предикатной схемой в базисе $\Pi$} назовем помеченный
неориентированный двудольный граф следующей структуры:

\begin{itemize}
\item каждая вершина из первой доли помечена некоторым символом из алфавита X или 
множеством символов из алфавита Y; 

\item каждая вершина второй доли помечена некоторым символом $\pi_i$ из множества $\Pi$ и 
соединена k ребрами, пронумерованными числами 1, ..., k, с вершинами из первой доли.
\end{itemize}
Вершины из первой доли будем называть узлами схемы, а вершины из второй доли ~-- её предикатными элементами. 
Узлы схемы, соединенные ребрами с предикатным элементом, будем называть полюсами этого элемента, 
а узлы, соответствующие входным переменным, ~-- полюсами схемы.
При этом считается, что узел является j-м полюсом предикатного элемента и соответствует 
его j-ой переменной, если соединяющее их ребро имеет номер j. Полюс схемы, которому приписано 
более одной входной переменной, называется кратным полюсом этой схемы. 
Будем считать элементарной такую предикатную схему, которая состоит либо из изолированной полюсной вершины, 
либо только из одного предикатного элемента $\pi_i$, 1 < i < k, где k ~-- число полюсов указанного элемента.

В тех случаях, когда это не вызывает разночтений, 
не будем различать узел схемы и переменную, 
символ которой приписан данной вершине, а также предикатный элемент и сам предикат, отвечающий этому элементу. 
Также, для удобства, в некоторых случаях будем использовать упрощенное описание предикатной схемы, 
опуская пометки дуг и некоторых внутренних вершин. 


% Список цитируемой литературы
\clearpage
\addcontentsline{toc}{section}{Список цитируемой литературы}
\thebibliography{99}
\RBibitem{Shu09}
    \by М.~С.~Шуплецов
    \paper Оценки высокой степени точности для сложности предикатных схем в~некоторых базисах
    \inbook Физико-математические науки
    \serial Уч\"eн. зап. Казан. гос. ун-та. Сер. Физ.-матем. науки
    \yr 2009
    \vol 151
    \issue 2
    \pages 173--184
    \publ Изд-во Казанского ун-та
    \publaddr Казань
    \mathnet{http://mi.mathnet.ru/uzku760}

\bibitem{Shu11}Методы синтеза и оценки сложности схем, построенных из элементов предикатного типа, диссертация

\bibitem{CSP10} Handbook of Constraint Programming, ISBN 9780444527264; 2010 г.
\endthebibliography

\end{document}
